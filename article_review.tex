Д.В. Нестеров\^{1}, Е.В. Розенгауз^{1,2}
\section*{Современное состояние лучевой диагностики рака поджелудочной железы (обзор литературы)} \label{pancreas_radiology}
%<*intro>
1ФГУ «Российский Научный Центр Радиологии и Хирургических Технологий», Санкт-Петербург
2ГБОУ ВПО «Северо-Западный государственный медицинский университет им. И.И.Мечникова», Санкт-Петербург

\subsection*{Реферат}

В статье рассмотрена роль методов лучевой визуализации в выявлении и дифференциальном диагнозе рака поджелудочной железы. 
Освещена проблема диагностики рака поджелудочной железы на ранних стадиях, возможные пути её решения. 
В результате анализа литературы установлено, что выявление и дифференциальная диагностика мелких образований поджелудочной железы трудна и требует мультимодального подхода. Ни один из современных методов не может выполнять роль «золотого стандарта» выявления и дифференциального диагноза рака поджелудочной железы. 

Ключевые слова: рак поджелудочной железы, компьютерная томография, магнитно резонансная томография, позитроно-эмисионная томография, эндоскопическое ультразвуковое исследование.


\subsection*{Вступление}


В клинической практике диагноз рака поджелудочной железы на основании данных одного метода исследования устанавливается достаточно редко. Как правило, используют мультимодальный подход с применением методов лучевой, лабораторной и патоморфологической диагностики. В этом обзоре литературы мы, в первую очередь, старались оценить роль методов лучевой диагностики.

Другие методики рассмотрены в объёме, необходимом для получения общей картины диагностического процесса рака поджелудочной железы и демонстрации места методов лучевой диагностики. 
Мы не останавливались подробно на трансабдоминальном УЗИ. Роль этого метода подробно изучена в конце ХХ века и за последние 10 лет исследований диагностической эффективности трансабдоминального УЗИ практически не проводилось. 
Также подробно не рассматривалась ретроградная холангио-панкреатография (РХПГ).  В группу <<нелучевых>> методов РХПГ попала в связи с тем,  что во всех рекомендациях к диагностике рака поджелудочной железы %1. Cascinu S., Jelic S., On behalf of the ESMO Guidelines Working Group. Pancreatic cancer: ESMO Clinical Recommendations for diagnosis, treatment and follow-up \/\/ Annals of Oncology. 2009. Т. 20. № Supplement 4. С. iv37–iv40., Pancreatic Section of the British Society of Gastroenterology, Pancreatic Society of Great Britain and Ireland, Association of Upper Gastrointestinal Surgeons of Great Britain and Ireland, Royal College of Pathologists, Special Interest Group for Gastro-I. Guidelines for the management of patients with pancreatic cancer periampullary and ampullary carcinomas \/\/ Gut. 2005. Т. 54. № suppl 5. С. v1–v16.] 
РХПГ описывается преимущественно как метод лечения и один из способов получения биопсийного материала. Диагностические возможности собственно рентгенологической картины существенно ниже, чем \textbf{изображений получаемых при} магнитно-резонансной холангиопанкреатографии (МРХПГ). \textbf{МРХПГ стала стандартом в изучении протоковой системы поджелудочной железы и печени, хотя это признаётся не всеми авторами \cite{kabanov_2012}.}

\textbf{
Задачами лучевой диагностики рака поджелудочной железы являются: выявление, дифференциальный диагноз и оценка распространённости. Согласно результатам отдельных исследований 80-90х годов каждая методика превосходит прочие в их решении.
Даже методы лабораторной диагностики позволяют определить операбельность опухоли \cite{zhang_clinical_2008}. Вполне можно заключить, что выбор метода может быть обусловлен технологической оснащённостью и субъективными предпочтениями.} Более того, за последние десять лет качество исследований и технологические характеристики методов улучшились. 
%Конечно, никаких революционных изменений они притерпели, но вместе с изменением качества должны вырасти и без того высокие 
Следовательно, должны вырасти операционные характеристики \footnote{под операционными характеристиками понимаются показатели чувствительности, специфичности, прогностической ценности положительного результата, прогностической ценности отрицательного результата и др.}.

\textbf{С другой стороны это было время активного  внедрения новых революционных подходов и важных технологических изменений в имеющиеся. Многие исследования проводились с разными протоколами сканирования, что снижало сопоставимость их результатов. Внимание иследовательских групп было привлечено к обкатке протоколов и определению сферы применения новых возможностей визуализации. За последние 10 лет таких внедрений не происходило, эйфория от технологических нововведений прошла, для большинства случаев определены оптимальнвые протоколы сканирования, что позволило оценить эффекты современных методов на выживаемость, их экономическую эффективность. И хотя при поверхностном анализе отечественной и зарубежной литературы мы обнаружили более 50 литературных обзоров, их библиографические списки более чем на половину состояли из данных полученных в 90х годах.}

\textbf{Цель данной статьи определить современные взгляды  на проблему лучевой диагностики рака поджелудочной железы}

Для сравнения диагностической эффективности исследований и отдельных симптомов мы указываем их чувствительность и специфичность. Для более точного представления о реальных показателях диагностической эффективности мы их приводим вместе с 95\% доверительным интервалом (ДИ) в виде: <<Значение \% (нижняя граница 95\% ДИ - верхняя граница 95\% ДИ )>>. Если авторы не вычисляли показатели чувствительности и специфичности или доверительные интервалы, но необходимая для этого информация содержалась в работах, мы вычисляли их самостоятельно.  Вычисление производили в языке програмирования <<R>> с помощью пакета <<mada>>.
%</intro>
\if 0
\subsection*{Материалы и методы}
Стратегия поиска.
Поиск источников литературы осуществлялся в базе данных PubMed и eLibrary за период 1990-2014. Поисковые запросы которые использовались при поиске в базах данных представлены в таблице \ref{table:query}. Таким образом в базе PubMed найдено ..... статей, в базе eLibrary 330. Для дальнейшего анализа отбирались статьи в которых могли бы рассматриваться вопросы диагностики рака поджелудочной железы. Из анализа и были исключены статьи посвящённые исключительно вопросам лечения, оценке результатов и осложнений проведенного лечения, исключительно диагностике нейроэндокринных и кистозных опухолей поджелудочной железы. После этого этапа из PubMed осталось .... статей, из eLibrary 56. Полные тексты удалось найти для .... статей из PubMed и ... статей из eLibrary.  В ходе обзора литературы рассмотрены все источники из базы данных PubMed за 1990 -- 2014 годы и источники найденные в списках литературы изученных источников. Из обзора были исключены статьи о диагностике нейроэндокринных и кистозных формах рака поджелудочной железы. 

В первую очередь осуществлялся поиск по базе pubmed cо следующим запросом (pancreas[Title/Abstract] OR pancreatic[Title/Abstract]) AND( 	(tomography[All Fields]) 	OR 	(ct[All Fields] OR "computed tomograpy"[All Fields] OR "computer tomograpy"[All Fields]) 	OR 	("magnetic resonance imaging"[MeSH Terms] OR ("magnetic"[All Fields] AND "resonance"[All Fields] AND "imaging"[All Fields]) OR 	"magnetic resonance imaging"[All Fields] OR "mri"[All Fields]) 	OR 	("ultrasonography"[Subheading] OR "ultrasonography"[All Fields] OR "us"[All Fields]) 	OR 	("ca-19-9 antigen"[MeSH Terms] OR ("ca-19-9"[All Fields] AND "antigen"[All Fields]) OR "ca-19-9 antigen"[All Fields] OR "ca19 9"[All Fields]) ) AND( 	cancer OR carcinoma OR neoplasm ) NOT( 	neuroendocrine[Title] 	OR case[Title] OR cyst[Title] OR cysts[Title] )

Для поиска отечественных источников литературы мы использовали базу данных elibrary.ru с поисковым запросом <<рак поджелудочной железы>>.

 

\begin{table}[htbp]
        \caption{Запрос базы данных PubMed}
       \begin{flushleft}
        \begin{tabular}{p{2cm}|p{10cm}|p{3cm}}
        \hline
        База данных & Поисковые слова & Область поиска\\ \hline
        AND & pancreatic, pancreas & заголовок\\ \hline
        AND & cancer, neoplasma, carcinoma, adenocarcinoma  & заголовок, ключевые слова, тезисы \\ \hline
        AND & CT, MRI, US, PET, computer tomography, magnetic resonance imaging, positron emission tomography, ultrasound & заголовок, ключевые слова, тезисы\\ \hline
        NOT & neuroedocrine, case & заголовок\\ \hline
        \end{tabular}
        \end{flushleft}
        \label{table:query}
\end{table}
\fi
%Таким образом было найдено 10521 статей. Из них полные тексты на английском языке удалось получить для ... источников из ...., на русском языке для .... источников из .... . Для ... статей были доступны тезисы статей на английском языке в которых содержалась количественная информация об объеме выборки и диагностической эффективности методов. 

%Также был проведен поиск по спискам литературы из найденных источников.

%Критерии включения материалов в обзор литературы представлены в таблице \ref{table:inclusion_crit}.
%Обзоры литературы были включены и использованы для поиска  источников не найденных в базах данных [ЗАХАРОВА О.П., КАРМАЗАНОВСКИЙ Г.Г., ЕГОРОВ В.И. РАК ПОДЖЕЛУДОЧНОЙ ЖЕЛЕЗЫ: СОВРЕМЕННАЯ СКТ ДИАГНОСТИКА И ОЦЕНКА РЕЗЕКТАБЕЛЬНОСТИ (обзор литературы) // ДИАГНОСТИЧЕСКАЯ И ИНТЕРВЕНЦИОННАЯ РАДИОЛОГИЯ. 5. № 2. С. 2011.],

%В таблице указана \ref{table:review_scheme} количественная информация о найденных источниках.

%<
%\begin{table}[htbp]
%        \caption{Этапы поиска материалов для обзора литературы}
%        \begin{flushleft}
%        \begin{tabular}{c|c}
 %       \hline
%        Название этапа & число источников\\ \hline
 %       Поиск в Pubmed  \\ \hline
%        Найдено тезисов & \\ \hline
%        Найдено полнотекстовых статей & \\ \hline
%       Включено полнотекстовых статей & \\ \hline
%        Включено тезисов & \\ \hline
%        Поиск в eLibrary  \\ \hline
%        Всего & 4385\\ \hline
%1       Найдено тезисов & \\ \hline
%        Найдено полнотекстовых статей & 330\\ \hline
%        
%        Включено полнотекстовых статей & 55? \\ \hline
%        Включено тезисов & \\ \hline
%        Включено авторефератов & 11\\ \hline
%        Поиск по спискам литературы \\ \hline
%        Найдено тезисов & \\ \hline
%        Найдено полнотекстовых статей & \\ \hline
%        Включено полнотекстовых статей & \\ \hline
%        Включено тезисов & \\ \hline
%        \end{tabular}
%        \end{flushleft}
%        \label{query}
%\end{table}
%<*detection>

\subsection*{Выявление}

В практической работе рентгенолога задачи выявления и дифференциального диагноза разделить достаточно сложно. На подсознательном уровне оба процесса происходят одновременно.\textbf{ В этом разделе будут описаны результаты исследований с репрезентативной выборкой, в которых перед рентгенологом ставилась задача диагностирования рака поджелудочной железы у пациентов с ещё не установленными изменениями в поджелудочной железе. Исследования выполненные по модели <<случай-контроль>> будут освещены в разделе посвященном дифференциальному диагнозу.}

%\textit{При выборе метода первичной визуализации предпочтение отдаётся не только самому точному. Большое значение имеет воспроизводимость его результатов. Если раньше обеспечение аппаратами СКТ, МРТ не позволяло изх использовать в качестве первичного, сейчас, ситуация изменилась. Более того растёт оснащённость позитрон Ноэми силовыми томографии. Учитывая хорошую оснащённость аппаратурой и несомненную актуальность диагностики рака поджелудочной железы, желательно ранней }

Рак поджелудочной железы выявляют как с помощью визуализационных методов, так и методами лабораторной диагностики. В доступной нам литературе не удалось найти данных о частоте выявления рака поджелудочной железы с помощью УЗИ в клинической практике. Многие авторы указывают на трансабдоминальное УЗИ, как <<первый рубеж диагностики>> у пациентов с желтухой\cite{muniraj_pancreatic_2013, donofrio_imaging_2010}. Однако, операционные характеристики трансабдоминального УЗИ уступают СКТ и МРТ, а наличие газа в кишке значительно затрудняет осмотр поджелудочной железы. \textbf{Также от других методов УЗИ отличает невозможность ретроспективного анализа результатов исследования. Улучшить УЗ-визуализацию поджелудочной железы можно с помощью контрастных веществ, режима тканевой гармоники и эластографии \cite{kinney_evidence-based_2010}. Однако, доказательная база по этим подходам довольно бедна, а остальные недостатки затрудняют исследования в этой области.}

Ключевую роль в визуализации рака поджелудочной железы выполняют СКТ и МРТ. Эндоскопическое УЗИ (эндоУЗИ) с биопсией стало предпочтительным методом для установления гистологического диагноза до операции. В комбинации с методами лучевой диагностики оказались полезны лабораторные методы, в частности анализ онкомаркеров. С этой целью могут быть использованы СЕА, СА19-9, SPan-1, DUPAN-2, MIC-1, alpha4GnT, PAM4, степень метилирования  ДНК секрета поджелудочной железы и KRAS в кале. Самым широко применяемым из перечисленных является CA19-9 \cite{muniraj_pancreatic_2013}.
%CA19-9 это эпитоп  антигена Sialis Lewis, который отсутствует у 10\% лиц европеоидной рассы, что позволяет его использовать в качестве онкомаркера у этих лиц. % muniraj_pancreatic_2013


\subsubsection*{СКТ}
Уже при первом научном исследовании, посвященном роли СКТ, сделан вывод, что эта методика является оптимальной для исследования больных раком поджелудочной железы \cite{hessel_prospective_1982}. Высокие операционные характеристики теста, простота выполнения, безопасность, доступность, воспроизводимость и возможность ретроспективной экспертной оценки сделали СКТ основным методом диагностики этого заболевания \cite{kinney_evidence-based_2010}.


Визуализация рака поджелудочной железы основана на выявлении разницы денситометрической плотности опухоли и окружающей ткани, масс-эффекта, пакреатической и/или билиарной гипертензии. Различия в васкуляризации опухоли и паренхимы позволяют визуализировать опухоль как гиподенсное образование на изображениях, полученных после контрастного усиления \cite{brennan_comprehensive_2007}. 

\textbf{Необходимость внутривенного контрастирования сегодня не вызывает сомнений, но следует иметь в виду, что речь идёт только о болюсном введении контрастного вещества с помощью автоматического инъектора \cite{muniraj_pancreatic_2013}.} <<Ручное>> введение контрастного вещества не позволяет достичь нужной степени контрастного усиления и диагностическая ценность такого исследования лишь немного превышает диагностическую ценность бесконрастной СКТ \cite{karmazanovsky_2009}. Сканирование следует проводить в фазу наибольшего усиления паренхимы поджелудочной железы, которая наступает в промежутке от 30 до 70~с после начала введения контрастного вещества. Исследования, проведенные в панкреатическую фазу, обладают на 24\% большей чувствительностью по сравнению с проведенными в портальную и на 35\% в артериальную фазу \cite{boland_pancreatic-phase_1999}. 
%Как правило, рекомендуется задержка в 40~с после начала инъекции контратсного вещества или 20~с после появления контратсного вещества в аорте (т.е. при использовании Bolus Tracking методики). Время достижения пика накопления контратсного вещества паренхимой поджелудочной железы достаточно индивидуально, даже относительно появления контрастного вещества в аорте, диапазон значений составляет около 14~с \cite{li_low-dose_2014}. Эти значения получены при введении 50 мл контратсного вещества со скоростью 5.5 мл/с, что отличается от стандартного протокола контрастирования при исследовании органов брюшной полости (около 90 мл со скоростью около 4.5 мл/с). 

Использование контрастных веществ с высокой концентрацией йода позволяет достичь более высокой денситометрической плотности сосудов и паренхимы поджелудочной железы, тем самым облегчая визуализацию \cite{fukukura_pancreatic_2008}. По другим данным, явных преимуществ в отношении визуализации опухоли применение высококонтрастных веществ не имеет \cite{fenchel_effect_2004}.

Выполнение мультипланарных реконструкций не требует много времени и позволяет повысить диагностическую эффективность исследования \cite{mehmet_erturk_pancreatic_2006}. 
Роль криволинейных реконструкций и реконструкций в проекции минимальной интенсивности не совсем ясна. Их использование позволяет визуализировать протоковую систему с качеством, сопоставимым с ЭРХПГ  \cite{raptopoulos_multiplanar_1998,nino-murcia_multidetector_2001, gong_role_2004}. Однако процесс построения занимает достаточно много времени, связан с возможным искажением размеров и формы анатомических структур, увеличивает операторозависимость в процессе получения изображений. 

Рекомендуется выполнять исследование до стентирования общего желчного протока в связи с тем, что наличие стента вызывает появление артефактов \cite{wong_staging_2008}. Числовых данных о степени их влияния на диагностическую эффективность в доступной литературе не найдено. 

\textbf{В клинической практике} чувствительность СКТ в выявлении рака поджелудочной железы составляет 75–100\% , специфичность 70–100\% [7–14]. Выявление образований диаметром выше 2~см, как правило, не представляет проблем для СКТ, чувствительность метода в данном случае превышает 98\% [10]. Ложно-отрицательные результаты могут быть получены \textbf{ в редких случаях}. Например на фоне сочетания асцита и портального тромбоза \cite{dewitt_comparison_2004}.

Чувствительность СКТ рассчитанная для опухолей диаметром до 20~мм лежит между 18 и 78\%. Результаты определения чувствительности СКТ в выявлении мелких раков поджелудочной железы представлены в таблице \ref{tab:small_tumors}.

% Table generated by Excel2LaTeX from sheet 'Лист3'
\begin{table}[htbp]
  \centering
  \caption{Чувствительность различных методов в выявлении мелких аденокарцином поджелудочной железы}
    \begin{tabular}{rrrrrrr}
    Исследование & Размеры & СКТ   & МРТ   & ПЭТ   & эндоУЗИ & Число больных \\
    Okano et al., 2011 \cite{okano_18f-fluorodeoxyglucose_2011} & <20   & 40\%  & 0\%   & 100\% & -     & 5 \\
    Matsumoto et al., 2013\cite{matsumoto_18-fluorodeoxyglucose_2013} & <20   & -     & -     & 68  & -     & 16 \\
    Seo et al., 2008\cite{seo_contribution_2008} & <20   & -     & -     & 81  & -     & 16 \\
    Bronstein et al., 2004 \cite{bronstein_detection_2004} & <20   & 78    & -     & -     & -     & 18 \\
    Legmann et al., 1998  \cite{legmann_pancreatic_1998}& <15   & 67    & -     & -     & 100   & 6 \\
    Ichikawa et al., 1997 \cite{ichikawa_pancreatic_1997}& <20   & 58    & -     & -     & -     & 12 \\
    Rose et al., 1999 \cite{rose_18fluorodeoxyglucose-positron_1999}& <21   & 18    & -     & -     & -     & - \\
    Dewitt et al., 2004\cite{dewitt_comparison_2004} & <25   & 53    & -     & -     & 89    & 19 \\
    Yoon et al., 2011 \cite{yoon_small_2011} & <20   & 27    & -     & -     & -     & 59 \\
    Maguchi et al., 2006 \cite{maguchi_small_2006} & <20   & 43    & -     & -     & 95    & 21 \\
    \end{tabular}%
  \label{tab:small_tumors}%
\end{table}%


\textbf{В перечисленных работах нет подробного описания причин ложно отрицательных результатов. Не совсем ясно является ли это исключительно проблемой выявления или обусловлено сложностью трактовки выявленных маловыраженных изменений. В таком случае чувствительность и специфичность будут зависеть от прогностической ценности положительного результата в связи с психологическими особенностями диагностического процессва. Так или иначе}
около 5-10\% опухолей поджелудочной железы при контрастном усилении  по данным визуальной оценки изоденсны паренхиме \cite{izuishi_impact_2010, karmazanovsky_2009}.
Однако, если измерить денситометрическую плотность в проекции опухоли, то она может отличаться от паренхимы поджелудочной желзы.
В исследовании Prokesh et al. разница в денситометрической плотности между <<изоденсной>> опухолью и паренхимой составила $9.25\pm11.3 HU$ в панкреатическую фазу и $4.15\pm8.5 HU$ в портальную \cite{prokesch_isoattenuating_2002}. Для гиподенсных опухолей разница была $74.76\pm35.61 HU$. Есть мнение, что для распознавания образования разность денситометрической плотности между опухолью и паренхимой должна составлять не менее 10 HU \cite{baron_understanding_1994}. Исследований физиологии восприятия по этому вопросу в доступной нам литературе опубликовано не было. 
\textbf{И видимо, требуют уточнения в современных условиях, т.к. повысились детализация и качество  изображений. По крайней мере нам удавалось визуализировать мелкие опухоли плотность которых отличалась от паренхимы менее чем на 10 HU}

В связи с отсутствием когортных исследований, посвященных скринингу рака поджелудочной железы с помощью всего спектра методов лучевой диагностики, оценить реальную распространенность опухолей такого рода невозможно. В среднем размер изоденсивных опухолей измеренных на макропрепарате  составляет 30~мм (от 15 до 40~мм). Иногда такие опухоли не удается визуализировать даже на макропрепарате \cite{bronstein_detection_2004} . Гистологически изоденсные раки поджелудочной железы характеризуются более низкой клеточной плотностью, б\'{о}льшим числом ацинусов и меньшим некрозов, что расценивается как ранние изменения \cite{yoon_small_2011}. Заподозрить наличие опухоли в таких случаях можно по косвенным признакам: массэффекту, обрыву и расширению панкреатического протока, атрофии паренхимы поджелудочной железы дистальнее места обрыва протока. 

Прямая визуализация таких опухолей возможна с помощью МРТ или ПЭТ с $^{18}F$-ФДГ \cite{kim_visually_2010}. Иногда их удаётся визуализировать с помощью МРТ без контрастного усиления, используя только диффузионно взвешенные последовательности \cite{takakura_clinical_2011}. С другой стороны Диагностическая эффективность ПЭТ, возможно, мало отличается от CA19-9 \cite{izuishi_impact_2010}, а в ряде случаев при СКТ удаётся выявить больше мелких опухолей, чем при МРТ \cite{okano_18f-fluorodeoxyglucose_2011}. Таким образом, вопрос о рациональном алгоритме исследования пациентов с подозрением на мелкую опухоль по-прежнему остаётся открытым. Оптимальным подходом, на сегодняшний день, выглядит профессионально выполненное эндоУЗИ с аспирационной биопсией. При положительном результате эндоУЗИ и отрицательном результате биопсии исследование (включая биопсию) целесообразно повторить \cite{tamm_retrospective_2007}.

\subsubsection*{МРТ}
Исследование рекомендуется проводить на высокопольных томографах (> 1~T), с обязательным использованием фазированных поверхностных катушек, автоматических инъекторов для введения контрастного вещества, мощных градиентов и быстрых импульсных последовательностей для уменьшения артефактов, связанных с дыханием \cite{donofrio_imaging_2010}. Рак поджелудочной железы визуализируется как гетерогенно гипоинтенсивная опухоль на Т1 взвешенных изображениях и имеет достаточно вариабельную интенсивность на T2 взвешенных изображениях. При контрастном усилении характер накопления контрастного вещества соответствует гиповаскулярному паттерну \cite{sahani_radiology_2008}. Применение быстрых программ (например 15ти секундная FLAME) также позволяет получить <<многофазное>> изображение поджелудочной железы \cite{arablinskij_2010}. Паренхима поджелудочной железы, лежащая дистальнее опухоли, также может иметь гипоинтенсивный сигнал. Отграничить опухоль опухоль от непораженной паренхимы, как правило, можно на диффузионно взвешенных изображениях (ДВИ) \cite{kartalis_diffusion-weighted_2009}.

Данные мета-анализа Bipat и др. показали, что чувствительность МРТ в выявлении рака поджелудочной железы уступает СКТ \cite{bipat_ultrasonography_2005}. Ряд более поздних исследований демонстрирует сопоставимые с СКТ диагностические характеристики \cite{fusari_comparison_2010, takakura_clinical_2011}. 

Описывается ряд сценариев, когда МРТ позволяет более уверенно трактовать природу изменений, например в случае изменения контура поджелудочной железы на фоне её очаговой жировой дистрофии \cite{miller_mri_2006, kim_focal_2007}. 

Качество визуализации вирсунгова протока при МР холангио-панкреатографии соответствует ЭРХПГ \cite{wehrmann_magnetic_2013} и может быть улучшено путём введения секретина \cite{matos_magnetic_2006}. 
 
 
Применение 3T МРТ, в совокупности с контрастным усилением, также не позволило добиться значимого увеличения чувствительности \cite{koelblinger_gadobenate_2011}. \textit{Для первичной диагностики солидных образований МРТ применяют при невозможности выполнения СКТ или необходимости виуализации протоковой системы \cite{kinney_evidence-based_2010}.}


\subsubsection*{ПЭТ}
Рак поджелудочной железы визуализируется как очаг накопления $^{18}F$-ФДГ. При этом SUV составляет 2-10. 

Ряд исследований продемонстрировали удивительные возможности ПЭТ в качестве инструмента выявления рака поджелудочной железы. Специфичность и чувствительность оцениваются близкими к 100\% и могут быть повышены с помощью отсроченного сканирования \cite{tlostanova_2008}. Сравнительные исследования также демонстрируют превосходство ПЭТ над другими методами. 
Безусловно, эти результаты наводят на мысль о применении ПЭТ в качестве самостоятельного и предпочтительного первого метода у пациентов с подозрением на рак поджелудочной железы.
Однако, проведенные за последние 10 лет мета-анализы продемонстрировали, что чувствительность и специфичность ПЭТ сопоставимы с другими методами (Таблица~\ref{tab:pet_metas}). 


% Table generated by Excel2LaTeX from sheet 'Лист2'
\begin{table}[htbp]
  \centering
  \caption{Операционные характеристики ПЭТ}
    \begin{tabular}{rrr}
          & Чувствительность & Специфичность \\ \hline
    Wang и др. 2013 \cite{wang_fdg-pet_2013} & 91\% (88\%-93\%) & 81\% (75\%-85\%) \\ \hline
    Rijkers и др., 2014\cite{rijkers_usefulness_2014} & 90\% (86\%-93\%) & 76\% (69\%-82\%) \\ \hline
    Tang и др., 2011\cite{tang_usefulness_2011} & 88\% (86\%-90\%) & 88\% (86\%-90\%) \\ \hline
    Orlando и др., 2004 \cite{orlando_meta-analysis:_2004} & 81\% (72\%-88\%) & 66\% ( 53\%-77\%) \\ \hline
    Wu и др., 2012\cite{wu_diagnostic_2012} (PET/CT) & 87\% (82\%-91\%) & 83\% (71\%-91\%) \\ \hline
    \end{tabular}%
  \label{tab:pet_metas}%
\end{table}%



Применение ПЭТ может быть целесообразно \textbf{при подозрении на опухоль} и отрицательных результатах СКТ. В этих случаях чувствительность ПЭТ составляет 73\% \cite{orlando_meta-analysis:_2004}.

Низкое пространственное разрешение и невозможность точной локализации очагов накопления не позволяют использовать ПЭТ в качестве самостоятельного метода. С одной стороны, проблема решается ПЭТ-КТ. Её операционные характеристики выше чем ПЭТ и могут быть повышены путём применения йод-содержащих контрастных веществ\cite{buchs_value_2011,wu_diagnostic_2012}. С другой стороны, возможность использования информации  о результатах предыдущих исследований у данного больного повышает диагностическую эффективность ПЭТ и приближает её к ПЭТ/КТ \cite{tang_usefulness_2011}. В большинстве случаев выполнение ПЭТ-КТ с целью выявления опухоли, как правило, не имеет преимуществ перед выполнением ПЭТ и СКТ отдельно.


Визуализация может быть значительно затруднена при малых размерах опухоли и ее расположении вблизи участков высокого физиологического накопления $^{18}F$-ФДГ. Примером такого рода опухолей являются ампулярные опухоли обычно имеющие малые размеры и расположенные вблизи стенки кишки [25].
Можно выделить две основные причины ложноотрицательных результатов при ПЭТ: гипергликемия и ранние стадии рака поджелудочной железы. По данным мета-анализа Orlando и др. у пациентов с гипергликемией чувствительность ПЭТ падает с 92\% до 88\% \cite{orlando_meta-analysis:_2004}.
Несмотря на множество технических улучшений, за последние годы диагностическая эффективность ПЭТ и ПЭТ/КТ как минимум не изменилась, возможно даже снизилась \cite{rijkers_usefulness_2014}. 
По мнению авторов эти результаты не объясняются различиями методологии исследований и, по всей видимости, являются следствием патофизиологических особенностей рака поджелудочной железы \cite{rijkers_usefulness_2014}. Маловероятно, что увеличение разрешающей способности значительно повысит диагностичекую эффективность метода. %не помню откуда я это взял, скорее всего просто умозаключение исходящее из того, что предыдущее повышение разрешающей способности не привело к повышению точности \cite{rijkers_usefulness_2014}


\subsubsection*{Эндо-УЗИ}


Одной из наиболее чувствительных методик в выявлении рака поджелудочной железы является эндоУЗИ. Данные о превосходстве  эндоУЗИ над другими методами в визуализации головки поджелудочной железы были получены достаточно давно \cite{yasuda_diagnosis_1988}. 


Чувствительность метода составляет 91-100\% даже в отношении образований диаметром менее 2~см. Этот метод позволяет напрямую визуализировать опухоли диаметром 2 мм \cite{helmstaedter_pancreatic_2008}. Применение эндоУЗИ с биопсией в качестве первого метода у пациентов с желтухой по экономической эффективности превосходит ЭРХПГ с \textbf{браш-биопсией} и СКТ с последующим эндоУЗИ \cite{chen_cost-minimization_2004}.


Прогностическая значимость положительного теста находится в пределах 92-98\%, прогностическая значимость отрицательного теста в переделах 90-100\%. Вместе с тем, специфичность исследования низка \cite{chang_role_2009} % ссылки 7-12
Этой позиции противоречат данные Tang и др., которые оценивают эндоУЗИ как высокоспецифичный метод, чувствительность которого ниже, чем ПЭТ \cite{tang_usefulness_2011}. Визуализация опухолей с помощью эндоУЗИ затруднена у пациентов с хроническим панкреатитом, из-за кальцинатов и изменений окружающей опухоль паренхимы.

У пациентов с неопределёнными результатами СКТ чувствительность и специфичность эндоУЗИ с биопсией составляли 87\% и 98\%, соответственно \cite{wang_use_2013}.

Использованию методу на первых рубежах диагностики мешают стандартные недостатки УЗИ -- сложность выполнения, операторозависимость, невозможность экспертной оценки \cite{fujii_pitfalls_2014}.





%</detection>
%<*evaluation>
\subsection*{Дифференциальный диагноз}

Дифференцировать аденокарциному поджелудочной железы следует от других ее гиповаскулярных солидных образований. Дифференциальный диагноз между различными гистологическими вариантами аденокарциномы представляет скорее академический интерес и не влияет на тактику лечения. С точки зрения определения тактики лечения наиболее важно отличить рак поджелудочной железы от очаговых форм хронического псевдотуморозного и аутоимунного панкреатита.

Мультимодальный подход, включающий лучевые и лабораторные методы, позволяет достаточно точно дифференцировать локальные формы аутоимунного панкреатита. \textbf{ К признакам аутоимунного панкреатита относится стеноз сегмента Вирсунгова протока протяженностью свыше 3 см, при диаметре супрастенотической части Вирсунгова протока не выше 6 мм, отсутствие атрофии паренхимы поджелудочной железы дистальнее места обструкции [42].} Ведущую роль в этой клинической ситуации играют методы лабораторной диагностики, в частности определение уровня IgG и IgG4. Однако, Kim et al. отмечают, что иногда они могут быть повышены у пациентов с изоденсным раком поджелудочной железы \cite{kim_visually_2010}. Недостаточное число публикаций по этой тематике не позволяет сделать выводы о распространенности проблемы.

Очаговые формы хронического панкреатита гораздо сложнее дифференцируются с аденокарциномой. Очаги визуализируются как гипо-/изоденсные образования при СКТ и гипоинтенсивные образования при МРТ [39–42]. Так как очаговый панкреатит является очагом воспаления, степень накопления глюкозы в нём выше, чем в окружающей паренхиме, что затрудняет дифференциальный диагноз с помощью ПЭТ [43]. 

Достаточно полезно определение уровня СА19-9. Метод прост в использовании и при дифференциальном диагнозе обладает чувствительностью до 90~\% \cite{klimov_2006}.
Чувствительность и специфичность зависят от выбранного порога.
Однако, ряд авторов не нашли статистически значимых различий в концентрации СА19-9 среди больных раком поджелудочной железы и хроническим псевдотуморозным панкреатитом \cite{kato_limited_2013}. Также ложноположительные результаты возможны при опухолях других локализаций \cite{herreros-villanueva_molecular_2013}

Симптомами, характерными для панкреатита, являются умеренная атрофия тела железы, постепенное сужение вирсунгова протока, визуализация не расширенных протоков, проходящих сквозь опухоль, (“duct penetrating sign”), неровность контуров протока, кальцификаты в поджелудочной железе. В свою очередь, для рака поджелудочной железы характерна резкая обструкция протока поджелудочной железы с атрофией ее паренхимы дистальнее места обструкции \cite{ichikawa_duct-penetrating_2001}.
Возможно, значение имеет не только расширение обоих протоков как таковое, но и характер их обструкции. %надо найти ссылку на статью, где утверждат что имеет
Симптом может быть обнаружен с помощью СКТ, МРТ и ЭРХПГ. Исследование 1982 года показало, что ЭРХПГ картина  расширения протоков не специфична (Таблица \ref{tab:double_duct}). 

% Table generated by Excel2LaTeX from sheet 'Лист4'
\begin{table}[htbp]
  \centering
  \caption{Диагностическая эффективность симптома <<двустволки>> по данным Plumley et al., 1982\cite{plumley_double_1982}}
    \begin{tabular}{p{6 cm}сс}
     Симптом & Чувствительность & Специфичность  \\ \hline
     Неровные контуры на уровне расширения как минимум одного из протоков. расширенных протоков & 61 (48-74) & 31 (18-49) \\ \hline
     Резкая обструкция одного из протоков & 76(62-86) & 68(51-81)  \\ \hline
     Резкая обструкция холедоха & 81(61-92) & 85(62-96)  \\ \hline
     Резкая обструкция Вирсунга & 71(51-86) & 50(28-72)  \\ \hline
    \multicolumn{4}{p{15cm}}{В скобках указан 95\% доверительный интервал}
    
    \end{tabular}%
    
  \label{tab:double_duct}%
\end{table}%



\subsubsection*{Повышение концентрации контрастного вещества в образовании}
Гиповаскулярные образования поджелудочной железы не имеют четкого паттерна контрастирования, позволяющего их дифференцировать. Если речь идёт о крупных образованиях, очевидной опухолевой инвазии сосудов, метастатическом поражении печени, проблем с дифференциальным диагнозом нет. Дискуссии вызывает дифференциальный диагноз маленьких образований или диффузных изменений поджелудочной железы.



Некоторые исследования продемонстрировали потенциальную возможность дифференциального диагноза на основании динамики накопления контрастного вещества. Анализ кривых «интенсивность сигнала/время» является полезным подходом, т.к. аденокарцинома имеет более поздний пик накопления по сравнению с псевдотуморозным панкреатитом (медленный подъем до пика на 180 секунде был характерен для аденокарцином). Yamada et al. продемонстрировали возможность дифференциального диагноза между двумя патологиями на основании анализа кривых «концентрация/время» при СКТ. При псевдотуморозном панкреатите авторы определяли отсроченное вымывание контрастного вещества, в то время как при раке поджелудочной железы наблюдали постепенное увеличение плотности с пиком после 150 секунды \cite{yamada_pancreatic_2010}.


\subsubsection*{Диффузионного взвешенные изображения}
Диффузионно взвешенные МРТ также являются многообещающим подходом в дифференциальном диагнозе \cite{wu_value_2012}. Опухоли, для которых характерно снижение диффузии в связи с высокой плотностью клеток, имеют сигнал высокой интенсивности и более низкий кажущийся коэффициент диффузии (ККД) по сравнению с нормальной тканью. В исследовании диффузионно взвешенной МРТ 38 пациентов, Fattahi et al. обнаружили, что при значении «b» 600 sec/mm2, псевдотуморозный панкреатит не был отличим от ткани поджелудочной железы, в то время как аденокарциномы имели гиперинтенсивный сигнал. Средний ККД для аденокарциномы ($1.46\pm0.18 *10-3 mm^2/sec$) был значимо ниже, чем его значение для псевдотуморозного панкреатита ($2.09\pm0.18* 10-3 mm^2/sec$) или нормальной паренхимы ($1.78\pm0.07 * 10-3 mm^2/sec$) \cite{fattahi_pancreatic_2009}.
Вместе с тем, диффузионно взвешенные изображения не рационально использовать для прямой визуализации образований в связи с тем, что примерно в половине случаев опухоль и дистальная часть паренхимы имеют гиперинтенсивный сигнал [48].

Добавление ДВИ позволяет достичь специфичности в дифференциальном диагнозе 91\%~(95\%~ДИ:71\%-98\%) превышающей ПЭТ/КТ \cite{wu_diagnostic_2012}. 
Предметом дискуссий остаётся пороговое значение <<b>> коэффициента, т.к. при его низких значениях ККД может быть повышен в связи с движением жидкости в капиллярах, а высокие значения снижают отношение сигнал/шум, требуют более длительного времени сканирования. 
Опубликованы данные о хороших дифференциально диагностических возможностях при <<b>> как 500 с/мм\textsuperscript{2} \cite{kartalis_diffusion-weighted_2009}, так и 1000 с/мм\textsuperscript{2} \cite{matsuki_diffusion-weighed_2007, ichikawa_high-b_2007}.


\subsubsection*{ПЭТ}

Злокачественные опухоли и очаги воспаления обладают высоким метаболизмом глюкозы, различить их с помощью ПЭТ достаточно трудно при любой локализации изменений. Если расценивать как опухоль любой очаг повышенного накопления глюкозы в поджелудочной железе, то специфичность метода составит 21~\%,~Таблица~\ref{tab:pet_pan_vs_pc}.

Для улучшения дифференциально-диагностических возможностей необходимо оценивать максимальный стандартизированный уровень захвата (SUVmax \footnote{от standardized uptake value}). Хотя SUVmax при очаговом панкреатите и раке поджелудочной железы статистически значимо различаются, определить пороговое значение SUVmax для разграничения двух патологий не удаётся \cite{kato_limited_2013}. По всей видимости, уверенно диагностировать рак поджелудочной железы можно при  значениях SUVmax выше 7,7 через 1 час и  9,98 через 2 часа после инъекции, а хронический псевдотуморозный панкреатит при SUVmax ниже 3,37 через 1  час и 3,53 через 2 часа \cite{kato_limited_2013}.

% latex table generated in R 3.1.0 by xtable 1.7-3 package
% Tue May 13 21:27:09 2014
\begin{table}[htbp]
\centering
\caption{Диагностическая эффективность ПЭТ в дифференциальном диагнозе рака поджелудочной железы }
\begin{tabular}{p{6cm}rr}
  \hline
 & Чувствительность, \% & Специфичность, \% \\ 
  \hline
Matsumoto et al., 2013 \cite{matsumoto_18-fluorodeoxyglucose_2013}* & 94 (90 - 96) & 21 (6 - 47) \\ 
  Van Kouwen et al., 2005 \cite{ van_kouwen_fdg-pet_2005 }* & 92~(78~-~99) & 87~(79~-~93) \\ 
  Rose et al., 1999   \cite{ rose_18fluorodeoxyglucose-positron_1999}* & 92~(83~-~97) & 85~59~-~97) \\ 
  Nakamoto et al.,2000 \cite{ nakamoto_delayed_2000} Пороговое значение SUV через 1~ч - 2.8 & 96~(84~-~100) & 75~(54~-~90) \\ 
  Nakamoto et al.,2000 \cite{ nakamoto_delayed_2000} Пороговое значение SUV через 2~ч - 2.4 & 100~(89~-~100) & 75~(54~-~90) \\ 
  Nakamoto et al.,2000 \cite{ nakamoto_delayed_2000} Пороговое значение SUV через 2~ч - 2.4 + изменение SUV на 15\% & 100~(89~-~100) & 80~(60~-~93) \\ 
  Ozaki et al., 2008 \cite{ ozaki_differentiation_2008}* & 73~(55~-~87) & 0~(0~-~18) \\ 
  Heinrich et al., 2005 \cite{ heinrich_positron_2005}* & 89~(78~-~96) & 69~(43~-~89) \\ 
  Lee et al., 2009 \cite{ lee_utility_2009}*,** & 82~(76~-~87) & 0~(0~-~16) \\ 
  Lee et al., 2009 \cite{ lee_utility_2009}*** & 97~(93~-~99) & 53~(31~-~74) \\ 
  Kato et al., 2013 \cite{ kato_limited_2013} повышение SUV& 67~(50~-~81) & 36~(14~-~65) \\ 
  Nakamoto et al.,2000\cite{ nakamoto_delayed_2000} повышение SUV & 81~(65~-~92) & 85~(66~-~96) \\ 
   \hline
   \multicolumn{ 3}{p{15cm}}{*-визуальный анализ накопления,**-сравнение с аутоимунным панкреатитом, ***-локальное накопление расценивалось как рак поджелудочной железы, диффузное как аутоимунный панкреатит}
   
\end{tabular}
\label{tab:pet_pan_vs_pc}
\end{table}

Возможным решением является оценка накопления $^{18}F$-ФДГ в динамике. Большинство исследований демонстрируют увеличение SUV в период от 1 до 2 часов после инъекции $^{18}F$-ФДГ у больных раком поджелудочной железы, в то время как в доброкачественных образованиях SUV уменьшается [60]. Использование этого подхода позволяет повысить диагностическую точность. В исследовании Тлостановой М.С. и др. чувствительность и специфичность ПЭТ были повышены с с 97\% и 95\% до 100\% \cite{tlostanova_2008}.

Не ясна зависимость SUV от размеров опухоли \cite{izuishi_impact_2010, seo_contribution_2008}. \textbf{ Наличие такой зависимости, с одной стороны, может объяснить данные о специфичности ПЭТ (21\%)  \cite{matsumoto_18-fluorodeoxyglucose_2013}, и ставит под вопрос целесообразность оценки SUVmax у пациентов с малыми размерами очага.}

Увеличение разрешающей способности ПЭТ за счёт коррекции аттенюации вряд ли повышает её дифференциально диагностические возможности \cite{kato_limited_2013}, вместе с тем сама коррекция может повлиять на измеряемый SUV \cite{kinahan_x-ray-based_2003}

Ложно положительные результаты ПЭТ могут наблюдаться у пациентов с ретроперитонеальным фиброзом, разделенной поджелудочной железой и сопутствующим панкреатитом, после установки назобилиарного катетера, тромбоза портальной вены, кровоизлияния в псевдокисту поджелудочной железы, постлучевого панкреатита, в зонах недавнего оперативного вмешательства или биопсии, стентирования холедоха. \cite{serrano_role_2010} 
%ссылка на обзор, найти нормальную ссылку. 
В описанных случаях дифференциальный диагноз может быть выполнен с помощью других методов.
\subsubsection*{Эндо-УЗИ}

Чувствительность и специфичность эндоУЗИ в отношении дифференциального диагноза рака поджелудочной железы и панкреатита составляет 81\% (79-84) и 93 (91-95) \cite{tang_usefulness_2011}. \textbf{ЭндоУЗИ позволило добиться 91\% чувствительности и 86\% специфичности в группе пациентов с гистологически верифицированным пакреатитом, ошибочно прооперированным на основании результатов СКТ  \cite{varadarajulu_role_2010}.}

Наличие хронического панкреатита снижает диагностические характеристики эндоУЗИ \cite{tang_usefulness_2011}. \textbf{ К недостаткам исследования можно отнести невозможность оценки вторичного поражения печени. Зачастую дифференциальный диагноз может быть выставлен на основании именно этого <<симптома>>. 
%Также высокая разрешающая способность и динамичность исследования делают эндо-УЗИ единственным методом лучевой диагностики дифференциальный диагноз в котором производится с помощью ПоЧиФоРа ИнРиКоС. 
}

Хотя некоторые исследователи и рекомендуют использование эндоУЗИ в качестве высокоспецифичного метода  \cite{tang_usefulness_2011}, большинство же склоняется что исключительно УЗ-картина опухоли и очагового панкреатита практически не отличается.%ссылка

Проведение при эндоУЗИ тонкоигольной биопсии позволяет повысить чувствительность до 94-100\% \cite{chang_role_2009} %8,10-14
Несмотря на инвазивность, процедура довольно безопасна -- осложнения возникают в 2\% случаев \cite{eloubeidi_prospective_2007}.

Несмотря на свои преимущества, биопсия не имеет абсолютной чувствительности. Доля ложно отрицательных результатов составляет 15-20\%. Они возникают по причине сопутствующего панкреатита или фиброза, неверного выбора места биопсии, аспирации крови, ошибочной интерпретации полученного материала.

Повысить дифференциально диагностическую ценность эндоУЗИ может компьютер-ассистированная обработка результатов \cite{zhu_differentiation_2013}, эластография \cite{ying_clinical_2013, xu_endoscopic_2013}


%\subsubsection{ПКТ}
%\subsubsection{Патологоанатомические методы}
%</evaluation>

%\subsection{Оценка распространённости}



%\subsubsection{СКТ}

%Множество авторов считают СКТ методом выбора для оценки сосудистой инвазии [
%ЗАХАРОВА О.П., КАРМАЗАНОВСКИЙ Г.Г., ЕГОРОВ В.И. РАК ПОДЖЕЛУДОЧНОЙ ЖЕЛЕЗЫ: СОВРЕМЕННАЯ СКТ ДИАГНОСТИКА И ОЦЕНКА РЕЗЕКТАБЕЛЬНОСТИ (обзор литературы) // ДИАГНОСТИЧЕСКАЯ И ИНТЕРВЕНЦИОННАЯ РАДИОЛОГИЯ. 5. № 2. С. 2011.]

%\subsubsubsection{[ЗАХАРОВА О.П., КАРМАЗАНОВСКИЙ Г.Г., ЕГОРОВ В.И. РАК ПОДЖЕЛУДОЧНОЙ ЖЕЛЕЗЫ: СОВРЕМЕННАЯ СКТ ДИАГНОСТИКА И ОЦЕНКА РЕЗЕКТАБЕЛЬНОСТИ обзор литературы // ДИАГНОСТИЧЕСКАЯ И ИНТЕРВЕНЦИОННАЯ РАДИОЛОГИЯ. 5. № 2. С. 2011.]}

%Чувствительность и специфичность СКТ при определении инвазии артерий составляет 79\% и 99\% соответственно, вен - 92\% и 100\% соответственно.[1
%. Li H. и др. Pancreatic adenocarcinoma: the different CT criteria for peripancreatic major arterial and venous invasion // Journal of computer assisted tomography. 2005. Т. 29. № 2. С. 170.] 
%Обобщённый результат оценки чувствительности и специфичности по данным мета-анализа Z.Wen-Yi et al[8] составил 77\% и 82\% соответственно. " Сразвитием воозможностей и технологий за  СКТ в период 2004-2008 гг. значения чувствительности и специфичности этой методики в диагностике инвазии сосудов выросли соответственно до 85\% и 82\% "

%По сравнению с интраоперационными данными положительная прогностическая ценность СКТ может достигать 100\%. Однако, она значительно снижается (до 83\%) при сравнении с данными гистологического исследования удалённых макропрепаратов. [
%1. Olivié D. и др. Predicting resectability of pancreatic head cancer with multi-detector CT. Surgical and pathologic correlation // JOP. 2007. Т. 8. № 6. С. 753–8.]

%Хотя проведение СКТ связано с высокой лучевой нагрузкой считается, что польза от исследования перевешивает потенциальные риски [23]. Вместе с тем некоторые иследователи для снижение лучевой нагрузки при первичном исследовании рекомендуют проводить двухфазное исследование, а при послеоперационном мониторинге только только одну фазу сканирования с КУ с толщиной среза 5-7 мм.

%\subsubsection{МРТ}

%\subsubsection{ПЭТ}


%\subsubsection{УЗИ}
%\subsubsection{ПКТ}




%\subsection{Магнитно резонанстная томография}



%<*conclusion1>
\subsection*{Заключение}

Диагностика рака поджелудочной железы на ранних стадия проблематична как для функциональных, так и для морфологических методов, и вряд ли может быть решена увеличением их разрешающей способности. 

Несмотря на значительный прогресс в методике выполнения исследований и интерпретации их результатов, существенный успех достигнут лишь для крупных опухолей.

Выявление и дифференциальная диагностика мелких образований поджелудочной железы трудны, требуют мультимодального подхода. Для многих наблюдений остаются справедливыми выводы первых исследований диагностики злокачественных образований поджелудочной железы говорящие о невозможности уверенного дифференциального диагноза.

Ни один из рассмотренных подходов не подходит на роль <<золотого стандарта>> выявления и дифференциального диагноза рака поджелудочной железы. По-прежнему требуется индивидуальный подход к каждому отдельному клиническому случаю. %Разрабатываются как методы решающие задачи в какой-либо конкретной клинической ситуации, так и универсальные методы общего назначения не оставляющие нерешенных вопросов в большинстве случаев.

%</conclusion1>